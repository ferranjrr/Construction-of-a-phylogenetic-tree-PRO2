Aquest programa modular ofereix un menú d\textquotesingle{}opcions per construir un arbre filogenètic basat en el mètode W\+P\+G\+MA. S\textquotesingle{}utilitzen les classes \hyperlink{class_especie}{Especie}, \hyperlink{class_cjt___especies}{Cjt\+\_\+\+Especies}, \hyperlink{class_cluster}{Cluster} i \hyperlink{class_cjt___clusters}{Cjt\+\_\+\+Clusters}.

Els nodes d\textquotesingle{}aquest arbre filogenètic, anomenats clústers, estan formats per espècies. Els clústers que estan dins de l\textquotesingle{}arbre estan a una certa distància entre ells (llargada de les branques de l\textquotesingle{}arbre) que és equivalent a la distància entre les espècies que estan dins d\textquotesingle{}aquests clústers.

Al principi del programa es llegeix un natural que servirà pel càlcul de distàncies. Després, es llegeix una sèrie d\textquotesingle{}instruccions que determinaran l\textquotesingle{}evolució del programa.

Aquest programa té diverses opcions, però la idea principal és\+: primer es creen unes certes espècies que els hi pots consultar el gen i la distància entre elles. Aquestes espècies estan dins d\textquotesingle{}un conjunt d\textquotesingle{}espècies que té associada una taula de distàncies entre elles i es pot imprimir. Convertint les espècies en clústers i passant-\/los al conjunt de clústers, es pot imprimir també una taula de distàncies entre els clústers del conjunt. Des d\textquotesingle{}allà, s\textquotesingle{}aplica l\textquotesingle{}algorisme W\+P\+G\+MA i s\textquotesingle{}imprimeix l\textquotesingle{}arbre filogenètic. 